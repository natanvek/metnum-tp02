\vspace{2em}

En este trabajo, habiendo implementado el método de la potencia y experimentado con él, observamos distintos problemas en el mismo, tales como el error numérico o su incapacidad de calcular los autovectores de las matrices con autovalores repetidos en valor absoluto. Es por esto que experimentamos con diferentes modificaciones al método propuesto para abarcar todos los casos posibles. Además, propusimos optimizaciones al algoritmo, como lo es ir de a pasos pares, para acortar la cantidad de iteraciones que requería para converger. 

\vspace{1em}
Luego, con la implementación del método, comenzamos a experimentar con distintos casos de uso práctico. Por un lado, en el apartado del \textit{Club de Karate}, vimos la utilidad de la centralidad de autovector y conectividad algebraica en la matriz laplaciana para evaluar las distintas características de una red. 

\vspace{1em}
Por otro lado, el análisis de las red `Ego' de Facebook nos permitió experimentar con diferentes aproximaciones de las mismas, evaluando sus similitudes a la original y determinando cuáles fueron las mejores. También, recurrimos al análisis de componentes principales (PCA) para reducir la dimensionalidad del conjunto de datos, y observamos los resultados sobre distintos rangos de la matriz de covarianza. Afirmamos de esta manera la posibilidad de aplicar una reducción, a bajo costo de calidad en nuestra predicción.

\vspace{1em}
En conclusión, observamos que el cálculo de autovalores y autovectores no es una tarea sencilla, pero a su vez posee una gran utilidad para el estudio de matrices. Nos brindan información de la estructura subayacente de los datos y su comportamiento, por lo que son imprescindibles para el análisis de toda clase de problemas.
